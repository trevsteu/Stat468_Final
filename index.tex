% Options for packages loaded elsewhere
\PassOptionsToPackage{unicode}{hyperref}
\PassOptionsToPackage{hyphens}{url}
\PassOptionsToPackage{dvipsnames,svgnames,x11names}{xcolor}
%
\documentclass[
  letterpaper,
  DIV=11,
  numbers=noendperiod]{scrreprt}

\usepackage{amsmath,amssymb}
\usepackage{iftex}
\ifPDFTeX
  \usepackage[T1]{fontenc}
  \usepackage[utf8]{inputenc}
  \usepackage{textcomp} % provide euro and other symbols
\else % if luatex or xetex
  \usepackage{unicode-math}
  \defaultfontfeatures{Scale=MatchLowercase}
  \defaultfontfeatures[\rmfamily]{Ligatures=TeX,Scale=1}
\fi
\usepackage{lmodern}
\ifPDFTeX\else  
    % xetex/luatex font selection
\fi
% Use upquote if available, for straight quotes in verbatim environments
\IfFileExists{upquote.sty}{\usepackage{upquote}}{}
\IfFileExists{microtype.sty}{% use microtype if available
  \usepackage[]{microtype}
  \UseMicrotypeSet[protrusion]{basicmath} % disable protrusion for tt fonts
}{}
\makeatletter
\@ifundefined{KOMAClassName}{% if non-KOMA class
  \IfFileExists{parskip.sty}{%
    \usepackage{parskip}
  }{% else
    \setlength{\parindent}{0pt}
    \setlength{\parskip}{6pt plus 2pt minus 1pt}}
}{% if KOMA class
  \KOMAoptions{parskip=half}}
\makeatother
\usepackage{xcolor}
\ifLuaTeX
  \usepackage{luacolor}
  \usepackage[soul]{lua-ul}
\else
  \usepackage{soul}
  
\fi
\setlength{\emergencystretch}{3em} % prevent overfull lines
\setcounter{secnumdepth}{5}
% Make \paragraph and \subparagraph free-standing
\ifx\paragraph\undefined\else
  \let\oldparagraph\paragraph
  \renewcommand{\paragraph}[1]{\oldparagraph{#1}\mbox{}}
\fi
\ifx\subparagraph\undefined\else
  \let\oldsubparagraph\subparagraph
  \renewcommand{\subparagraph}[1]{\oldsubparagraph{#1}\mbox{}}
\fi


\providecommand{\tightlist}{%
  \setlength{\itemsep}{0pt}\setlength{\parskip}{0pt}}\usepackage{longtable,booktabs,array}
\usepackage{calc} % for calculating minipage widths
% Correct order of tables after \paragraph or \subparagraph
\usepackage{etoolbox}
\makeatletter
\patchcmd\longtable{\par}{\if@noskipsec\mbox{}\fi\par}{}{}
\makeatother
% Allow footnotes in longtable head/foot
\IfFileExists{footnotehyper.sty}{\usepackage{footnotehyper}}{\usepackage{footnote}}
\makesavenoteenv{longtable}
\usepackage{graphicx}
\makeatletter
\def\maxwidth{\ifdim\Gin@nat@width>\linewidth\linewidth\else\Gin@nat@width\fi}
\def\maxheight{\ifdim\Gin@nat@height>\textheight\textheight\else\Gin@nat@height\fi}
\makeatother
% Scale images if necessary, so that they will not overflow the page
% margins by default, and it is still possible to overwrite the defaults
% using explicit options in \includegraphics[width, height, ...]{}
\setkeys{Gin}{width=\maxwidth,height=\maxheight,keepaspectratio}
% Set default figure placement to htbp
\makeatletter
\def\fps@figure{htbp}
\makeatother
% definitions for citeproc citations
\NewDocumentCommand\citeproctext{}{}
\NewDocumentCommand\citeproc{mm}{%
  \begingroup\def\citeproctext{#2}\cite{#1}\endgroup}
\makeatletter
 % allow citations to break across lines
 \let\@cite@ofmt\@firstofone
 % avoid brackets around text for \cite:
 \def\@biblabel#1{}
 \def\@cite#1#2{{#1\if@tempswa , #2\fi}}
\makeatother
\newlength{\cslhangindent}
\setlength{\cslhangindent}{1.5em}
\newlength{\csllabelwidth}
\setlength{\csllabelwidth}{3em}
\newenvironment{CSLReferences}[2] % #1 hanging-indent, #2 entry-spacing
 {\begin{list}{}{%
  \setlength{\itemindent}{0pt}
  \setlength{\leftmargin}{0pt}
  \setlength{\parsep}{0pt}
  % turn on hanging indent if param 1 is 1
  \ifodd #1
   \setlength{\leftmargin}{\cslhangindent}
   \setlength{\itemindent}{-1\cslhangindent}
  \fi
  % set entry spacing
  \setlength{\itemsep}{#2\baselineskip}}}
 {\end{list}}
\usepackage{calc}
\newcommand{\CSLBlock}[1]{\hfill\break\parbox[t]{\linewidth}{\strut\ignorespaces#1\strut}}
\newcommand{\CSLLeftMargin}[1]{\parbox[t]{\csllabelwidth}{\strut#1\strut}}
\newcommand{\CSLRightInline}[1]{\parbox[t]{\linewidth - \csllabelwidth}{\strut#1\strut}}
\newcommand{\CSLIndent}[1]{\hspace{\cslhangindent}#1}

\KOMAoption{captions}{tableheading}
\makeatletter
\@ifpackageloaded{bookmark}{}{\usepackage{bookmark}}
\makeatother
\makeatletter
\@ifpackageloaded{caption}{}{\usepackage{caption}}
\AtBeginDocument{%
\ifdefined\contentsname
  \renewcommand*\contentsname{Table of contents}
\else
  \newcommand\contentsname{Table of contents}
\fi
\ifdefined\listfigurename
  \renewcommand*\listfigurename{List of Figures}
\else
  \newcommand\listfigurename{List of Figures}
\fi
\ifdefined\listtablename
  \renewcommand*\listtablename{List of Tables}
\else
  \newcommand\listtablename{List of Tables}
\fi
\ifdefined\figurename
  \renewcommand*\figurename{Figure}
\else
  \newcommand\figurename{Figure}
\fi
\ifdefined\tablename
  \renewcommand*\tablename{Table}
\else
  \newcommand\tablename{Table}
\fi
}
\@ifpackageloaded{float}{}{\usepackage{float}}
\floatstyle{ruled}
\@ifundefined{c@chapter}{\newfloat{codelisting}{h}{lop}}{\newfloat{codelisting}{h}{lop}[chapter]}
\floatname{codelisting}{Listing}
\newcommand*\listoflistings{\listof{codelisting}{List of Listings}}
\makeatother
\makeatletter
\makeatother
\makeatletter
\@ifpackageloaded{caption}{}{\usepackage{caption}}
\@ifpackageloaded{subcaption}{}{\usepackage{subcaption}}
\makeatother
\ifLuaTeX
  \usepackage{selnolig}  % disable illegal ligatures
\fi
\usepackage{bookmark}

\IfFileExists{xurl.sty}{\usepackage{xurl}}{} % add URL line breaks if available
\urlstyle{same} % disable monospaced font for URLs
\hypersetup{
  pdftitle={Stat468\_Final},
  pdfauthor={Trevor Steunenberg},
  colorlinks=true,
  linkcolor={blue},
  filecolor={Maroon},
  citecolor={Blue},
  urlcolor={Blue},
  pdfcreator={LaTeX via pandoc}}

\title{Stat468\_Final}
\author{Trevor Steunenberg}
\date{2025-07-12}

\begin{document}
\maketitle

\renewcommand*\contentsname{Table of contents}
{
\hypersetup{linkcolor=}
\setcounter{tocdepth}{2}
\tableofcontents
}
\bookmarksetup{startatroot}

\chapter{Abstract}\label{abstract}

Relative to every other pick, much is each selection in the NHL Entry
Draft worth? This question is important because trades involving solely
draft picks are rather common (there were 18 this past year) and knowing
the relative value of draft picks allows teams to make better decisions
regarding these trades.

Depending on the context, there can be countless approaches one could
take to quantify the value of a draft pick. As one example, if
estimating the fairness of a trade, one could compare the assets given
up and acquired to previous trades. In contrast, this report will
estimate value of pick \(n\) by utilizing the points contributed and
games played by previous players selected at pick \(n\) along with a
\(k\)-nearest neighbours algorithm.

The data from this report is taken from
\href{https://www.hockey-reference.com/draft/}{HockeyReference}, which
has data on the NHL Draft and player games played and point share counts
dating back to 1963.

Some work in this area has been done before, such as:

\begin{itemize}
\item
  \href{https://summit.sfu.ca/_flysystem/fedora/2023-02/etd22223.pdf}{Valuation
  of NHL Draft Picks using Functional Data Analysis}
\item
  \href{https://soundofhockey.com/2022/06/06/examining-the-value-of-nhl-draft-picks/amp/}{Examining
  the value of NHL Draft picks}
\item
  \href{https://www.broadstreethockey.com/post/nhl-draft-pick-value-trading-up/}{NHL
  draft: What does it cost to trade up?}
\end{itemize}

As an interesting aside, Eric Tulsky, who wrote the last article listed
above in 2013, was hired as General Manager of the Carolina Hurricanes
in 2024. As can be seen by the charts presented in the resources above,
NHL draft picks do \ul{\textbf{not}} decrease in value linearly (ie the
difference in value pick 1 and 30 is much greater than between pick 101
and 130).

Since the NHL has changed dramatically over the years, care must be
taken to ensure we do not include drafts from too long ago. The primary
concern with including data from many years ago is that teams have
likely become better at evaluating prospects as more advanced statistics
have been developed, meaning that there are likely fewer late round
draft ``steals'' in the 2020s than there were in the 1980s. Thus
including drafts from the 1980s would skew our calculations because it
would overestimate contributions by players who were drafted in the
later rounds, since those players would potentially have been drafted
sooner if the teams of the 1980s had the resources available to teams
today. This would make our model a poor estimator of draft pick value
for drafts occurring in the 2020s. That being said, players drafted in
recent years have not had sufficient time to contribute to their teams,
so we should not include drafts from too recent either. Ideally, we
should wait until all players from a draft class have retired before
including it in our analysis . Practically speaking, this is not
feasible since players can have very long careers (for example, Alex
Ovechkin was drafted in 2004) which would force us to include older
drafts to maintain the same sample size, which is also unideal as
explained above.

\bookmarksetup{startatroot}

\chapter{Question}\label{question}

This is a placeholder for Ch0 -- Question.

\bookmarksetup{startatroot}

\chapter{1\_Import}\label{import}

This is a placeholder for Ch1 -- Import.

\bookmarksetup{startatroot}

\chapter{2\_Tidy}\label{tidy}

This is a placeholder for Ch2 -- Tidy.

\bookmarksetup{startatroot}

\chapter{3\_Transform}\label{transform}

This is a placeholder for Ch3 -- Transform.

\bookmarksetup{startatroot}

\chapter{4\_Visualize}\label{visualize}

This is a placeholder for Ch4 -- Visualize.

\bookmarksetup{startatroot}

\chapter{5\_Model}\label{model}

This is a placeholder for Ch5 -- Model.

\bookmarksetup{startatroot}

\chapter{6\_Communicate}\label{communicate}

This is a placeholder for Ch6 -- Communicate.

\bookmarksetup{startatroot}

\chapter*{References}\label{references}
\addcontentsline{toc}{chapter}{References}

\markboth{References}{References}

\phantomsection\label{refs}
\begin{CSLReferences}{0}{1}
\end{CSLReferences}



\end{document}
