% Options for packages loaded elsewhere
\PassOptionsToPackage{unicode}{hyperref}
\PassOptionsToPackage{hyphens}{url}
\PassOptionsToPackage{dvipsnames,svgnames,x11names}{xcolor}
%
\documentclass[
  letterpaper,
  DIV=11,
  numbers=noendperiod]{scrreprt}

\usepackage{amsmath,amssymb}
\usepackage{iftex}
\ifPDFTeX
  \usepackage[T1]{fontenc}
  \usepackage[utf8]{inputenc}
  \usepackage{textcomp} % provide euro and other symbols
\else % if luatex or xetex
  \usepackage{unicode-math}
  \defaultfontfeatures{Scale=MatchLowercase}
  \defaultfontfeatures[\rmfamily]{Ligatures=TeX,Scale=1}
\fi
\usepackage{lmodern}
\ifPDFTeX\else  
    % xetex/luatex font selection
\fi
% Use upquote if available, for straight quotes in verbatim environments
\IfFileExists{upquote.sty}{\usepackage{upquote}}{}
\IfFileExists{microtype.sty}{% use microtype if available
  \usepackage[]{microtype}
  \UseMicrotypeSet[protrusion]{basicmath} % disable protrusion for tt fonts
}{}
\makeatletter
\@ifundefined{KOMAClassName}{% if non-KOMA class
  \IfFileExists{parskip.sty}{%
    \usepackage{parskip}
  }{% else
    \setlength{\parindent}{0pt}
    \setlength{\parskip}{6pt plus 2pt minus 1pt}}
}{% if KOMA class
  \KOMAoptions{parskip=half}}
\makeatother
\usepackage{xcolor}
\ifLuaTeX
  \usepackage{luacolor}
  \usepackage[soul]{lua-ul}
\else
  \usepackage{soul}
  
\fi
\setlength{\emergencystretch}{3em} % prevent overfull lines
\setcounter{secnumdepth}{5}
% Make \paragraph and \subparagraph free-standing
\ifx\paragraph\undefined\else
  \let\oldparagraph\paragraph
  \renewcommand{\paragraph}[1]{\oldparagraph{#1}\mbox{}}
\fi
\ifx\subparagraph\undefined\else
  \let\oldsubparagraph\subparagraph
  \renewcommand{\subparagraph}[1]{\oldsubparagraph{#1}\mbox{}}
\fi

\usepackage{color}
\usepackage{fancyvrb}
\newcommand{\VerbBar}{|}
\newcommand{\VERB}{\Verb[commandchars=\\\{\}]}
\DefineVerbatimEnvironment{Highlighting}{Verbatim}{commandchars=\\\{\}}
% Add ',fontsize=\small' for more characters per line
\usepackage{framed}
\definecolor{shadecolor}{RGB}{241,243,245}
\newenvironment{Shaded}{\begin{snugshade}}{\end{snugshade}}
\newcommand{\AlertTok}[1]{\textcolor[rgb]{0.68,0.00,0.00}{#1}}
\newcommand{\AnnotationTok}[1]{\textcolor[rgb]{0.37,0.37,0.37}{#1}}
\newcommand{\AttributeTok}[1]{\textcolor[rgb]{0.40,0.45,0.13}{#1}}
\newcommand{\BaseNTok}[1]{\textcolor[rgb]{0.68,0.00,0.00}{#1}}
\newcommand{\BuiltInTok}[1]{\textcolor[rgb]{0.00,0.23,0.31}{#1}}
\newcommand{\CharTok}[1]{\textcolor[rgb]{0.13,0.47,0.30}{#1}}
\newcommand{\CommentTok}[1]{\textcolor[rgb]{0.37,0.37,0.37}{#1}}
\newcommand{\CommentVarTok}[1]{\textcolor[rgb]{0.37,0.37,0.37}{\textit{#1}}}
\newcommand{\ConstantTok}[1]{\textcolor[rgb]{0.56,0.35,0.01}{#1}}
\newcommand{\ControlFlowTok}[1]{\textcolor[rgb]{0.00,0.23,0.31}{#1}}
\newcommand{\DataTypeTok}[1]{\textcolor[rgb]{0.68,0.00,0.00}{#1}}
\newcommand{\DecValTok}[1]{\textcolor[rgb]{0.68,0.00,0.00}{#1}}
\newcommand{\DocumentationTok}[1]{\textcolor[rgb]{0.37,0.37,0.37}{\textit{#1}}}
\newcommand{\ErrorTok}[1]{\textcolor[rgb]{0.68,0.00,0.00}{#1}}
\newcommand{\ExtensionTok}[1]{\textcolor[rgb]{0.00,0.23,0.31}{#1}}
\newcommand{\FloatTok}[1]{\textcolor[rgb]{0.68,0.00,0.00}{#1}}
\newcommand{\FunctionTok}[1]{\textcolor[rgb]{0.28,0.35,0.67}{#1}}
\newcommand{\ImportTok}[1]{\textcolor[rgb]{0.00,0.46,0.62}{#1}}
\newcommand{\InformationTok}[1]{\textcolor[rgb]{0.37,0.37,0.37}{#1}}
\newcommand{\KeywordTok}[1]{\textcolor[rgb]{0.00,0.23,0.31}{#1}}
\newcommand{\NormalTok}[1]{\textcolor[rgb]{0.00,0.23,0.31}{#1}}
\newcommand{\OperatorTok}[1]{\textcolor[rgb]{0.37,0.37,0.37}{#1}}
\newcommand{\OtherTok}[1]{\textcolor[rgb]{0.00,0.23,0.31}{#1}}
\newcommand{\PreprocessorTok}[1]{\textcolor[rgb]{0.68,0.00,0.00}{#1}}
\newcommand{\RegionMarkerTok}[1]{\textcolor[rgb]{0.00,0.23,0.31}{#1}}
\newcommand{\SpecialCharTok}[1]{\textcolor[rgb]{0.37,0.37,0.37}{#1}}
\newcommand{\SpecialStringTok}[1]{\textcolor[rgb]{0.13,0.47,0.30}{#1}}
\newcommand{\StringTok}[1]{\textcolor[rgb]{0.13,0.47,0.30}{#1}}
\newcommand{\VariableTok}[1]{\textcolor[rgb]{0.07,0.07,0.07}{#1}}
\newcommand{\VerbatimStringTok}[1]{\textcolor[rgb]{0.13,0.47,0.30}{#1}}
\newcommand{\WarningTok}[1]{\textcolor[rgb]{0.37,0.37,0.37}{\textit{#1}}}

\providecommand{\tightlist}{%
  \setlength{\itemsep}{0pt}\setlength{\parskip}{0pt}}\usepackage{longtable,booktabs,array}
\usepackage{calc} % for calculating minipage widths
% Correct order of tables after \paragraph or \subparagraph
\usepackage{etoolbox}
\makeatletter
\patchcmd\longtable{\par}{\if@noskipsec\mbox{}\fi\par}{}{}
\makeatother
% Allow footnotes in longtable head/foot
\IfFileExists{footnotehyper.sty}{\usepackage{footnotehyper}}{\usepackage{footnote}}
\makesavenoteenv{longtable}
\usepackage{graphicx}
\makeatletter
\def\maxwidth{\ifdim\Gin@nat@width>\linewidth\linewidth\else\Gin@nat@width\fi}
\def\maxheight{\ifdim\Gin@nat@height>\textheight\textheight\else\Gin@nat@height\fi}
\makeatother
% Scale images if necessary, so that they will not overflow the page
% margins by default, and it is still possible to overwrite the defaults
% using explicit options in \includegraphics[width, height, ...]{}
\setkeys{Gin}{width=\maxwidth,height=\maxheight,keepaspectratio}
% Set default figure placement to htbp
\makeatletter
\def\fps@figure{htbp}
\makeatother
% definitions for citeproc citations
\NewDocumentCommand\citeproctext{}{}
\NewDocumentCommand\citeproc{mm}{%
  \begingroup\def\citeproctext{#2}\cite{#1}\endgroup}
\makeatletter
 % allow citations to break across lines
 \let\@cite@ofmt\@firstofone
 % avoid brackets around text for \cite:
 \def\@biblabel#1{}
 \def\@cite#1#2{{#1\if@tempswa , #2\fi}}
\makeatother
\newlength{\cslhangindent}
\setlength{\cslhangindent}{1.5em}
\newlength{\csllabelwidth}
\setlength{\csllabelwidth}{3em}
\newenvironment{CSLReferences}[2] % #1 hanging-indent, #2 entry-spacing
 {\begin{list}{}{%
  \setlength{\itemindent}{0pt}
  \setlength{\leftmargin}{0pt}
  \setlength{\parsep}{0pt}
  % turn on hanging indent if param 1 is 1
  \ifodd #1
   \setlength{\leftmargin}{\cslhangindent}
   \setlength{\itemindent}{-1\cslhangindent}
  \fi
  % set entry spacing
  \setlength{\itemsep}{#2\baselineskip}}}
 {\end{list}}
\usepackage{calc}
\newcommand{\CSLBlock}[1]{\hfill\break\parbox[t]{\linewidth}{\strut\ignorespaces#1\strut}}
\newcommand{\CSLLeftMargin}[1]{\parbox[t]{\csllabelwidth}{\strut#1\strut}}
\newcommand{\CSLRightInline}[1]{\parbox[t]{\linewidth - \csllabelwidth}{\strut#1\strut}}
\newcommand{\CSLIndent}[1]{\hspace{\cslhangindent}#1}

\KOMAoption{captions}{tableheading}
\makeatletter
\@ifpackageloaded{bookmark}{}{\usepackage{bookmark}}
\makeatother
\makeatletter
\@ifpackageloaded{caption}{}{\usepackage{caption}}
\AtBeginDocument{%
\ifdefined\contentsname
  \renewcommand*\contentsname{Table of contents}
\else
  \newcommand\contentsname{Table of contents}
\fi
\ifdefined\listfigurename
  \renewcommand*\listfigurename{List of Figures}
\else
  \newcommand\listfigurename{List of Figures}
\fi
\ifdefined\listtablename
  \renewcommand*\listtablename{List of Tables}
\else
  \newcommand\listtablename{List of Tables}
\fi
\ifdefined\figurename
  \renewcommand*\figurename{Figure}
\else
  \newcommand\figurename{Figure}
\fi
\ifdefined\tablename
  \renewcommand*\tablename{Table}
\else
  \newcommand\tablename{Table}
\fi
}
\@ifpackageloaded{float}{}{\usepackage{float}}
\floatstyle{ruled}
\@ifundefined{c@chapter}{\newfloat{codelisting}{h}{lop}}{\newfloat{codelisting}{h}{lop}[chapter]}
\floatname{codelisting}{Listing}
\newcommand*\listoflistings{\listof{codelisting}{List of Listings}}
\makeatother
\makeatletter
\makeatother
\makeatletter
\@ifpackageloaded{caption}{}{\usepackage{caption}}
\@ifpackageloaded{subcaption}{}{\usepackage{subcaption}}
\makeatother
\ifLuaTeX
  \usepackage{selnolig}  % disable illegal ligatures
\fi
\usepackage{bookmark}

\IfFileExists{xurl.sty}{\usepackage{xurl}}{} % add URL line breaks if available
\urlstyle{same} % disable monospaced font for URLs
\hypersetup{
  pdftitle={Stat468 Final Project},
  pdfauthor={Trevor Steunenberg},
  colorlinks=true,
  linkcolor={blue},
  filecolor={Maroon},
  citecolor={Blue},
  urlcolor={Blue},
  pdfcreator={LaTeX via pandoc}}

\title{Stat468 Final Project}
\author{Trevor Steunenberg}
\date{2025-07-12}

\begin{document}
\maketitle

\renewcommand*\contentsname{Table of contents}
{
\hypersetup{linkcolor=}
\setcounter{tocdepth}{2}
\tableofcontents
}
\bookmarksetup{startatroot}

\chapter{Index}\label{index}

\section{Abstract}\label{abstract}

In the days leading up to and days of the 2025 NHL Entry Draft there
were 18 trades which only included draft picks. This report aims to use
player contribution data to determine the relative value of selections
in the NHL Entry Draft. Knowing the relative value of picks allows NHL
teams to both determine whether they should accept trade offers they
have received as well as propose favourable trades to other teams.

Depending on the context, there can be countless approaches one could
take to quantify the value of a draft pick. As one example, one could
estimate the fairness of a trade by comparing the assets given up and
acquired to previous trades. In contrast, this report will estimate
value of pick \(n\) by utilizing the points contributed and games played
by previous players selected at pick \(n\) along with a \(k\)-nearest
neighbours algorithm.

The data used by this report is imported from
\href{https://www.hockey-reference.com/draft/}{HockeyReference}, which
has data on the NHL Draft and player games played and point share counts
dating back to 1963, though we will only use a subset of this data as
will be explained later.

Some work in this area has been done before, such as:

\begin{itemize}
\item
  \href{https://summit.sfu.ca/_flysystem/fedora/2023-02/etd22223.pdf}{Valuation
  of NHL Draft Picks using Functional Data Analysis}
\item
  \href{https://soundofhockey.com/2022/06/06/examining-the-value-of-nhl-draft-picks/amp/}{Examining
  the value of NHL Draft picks}
\item
  \href{https://www.broadstreethockey.com/post/nhl-draft-pick-value-trading-up/}{NHL
  draft: What does it cost to trade up?}
\end{itemize}

As an interesting aside, Eric Tulsky, who wrote the last article listed
above in 2013, was hired as General Manager of the Carolina Hurricanes
in 2024. One critical point that it is common knowledge in ice hockey
circles and confirmed by the resources above, along with the
\href{https://trevsteu.github.io/Stat468_Final/model.html}{Model
chapter} of this report is that NHL draft picks do \ul{\textbf{not}}
decrease in value linearly. In particular, the difference in value pick
1 and 30 is much greater than between pick 101 and 130.

Note that if picks did decrease linearly in value linearly then it would
be very easy to create a model of draft pick value since we would have

\[
v_1 = v_2 + c = v_3 + 2c = ... = v_{224} + 223c
\]

where \(c > 0\) and \(v_i\) is the value of the \(i^{\text{th}}\)
selection, meaning we would only have to find the value of \(c\).

\bookmarksetup{startatroot}

\chapter{Question}\label{question}

This report will estimate the relative value of selections in the NHL
Entry Draft. I am not sure what else I am supposed to put here.

\bookmarksetup{startatroot}

\chapter{Import}\label{import}

\section{Introduction}\label{introduction}

As mentioned before, we will be importing data from
\href{https://www.hockey-reference.com/draft}{Hockey Reference}. Before
we import any data, it's important to consider \emph{which} and
\emph{how many} years we want to include in this analysis. Since the NHL
has changed dramatically over the years, care must be taken to ensure we
do not include drafts from too long ago. The primary concern with
including data from too many years ago is that teams have likely changed
their drafting approach over time. For example, teams may have become
better at evaluating prospects as more advanced statistics have been
developed, meaning that there are likely fewer late round draft
``steals'' in the 2020s than there were in the 1980s. Thus including
drafts from the 1980s would skew our calculations because it would
overestimate contributions by players who were drafted in the later
rounds, since those players would potentially have been drafted sooner
if the teams of the 1980s had the resources available to teams today.
This would make our model a poor estimator of draft pick value for
drafts occurring in the 2020s. That being said, players drafted in
recent years have not had sufficient time to contribute to their teams,
so we should not include drafts from too recently either. Ideally, we
would wait until all players from a draft class have retired before
including it in our analysis . Practically speaking, this is not
feasible since players can have very long careers (for example, Alex
Ovechkin was drafted in 2004 and is still playing) which would force us
to include older drafts to maintain the same sample size, which is also
not ideal as explained above.

Having considered this, we make the somewhat arbitrary decision to use
the 25 drafts between and 1996 and 2020 (inclusive). Note that a
significant portion of the players in our dataset are still active, so
we will have to make an adjustment to account for this. Additionally, it
makes sense to give more recent drafts more weight for the reasons
described above. We will make both of these adjustments in the
\href{https://trevsteu.github.io/Stat468_Final/transform.html}{Transform
chapter}.

\section{Setup}\label{setup}

\begin{Shaded}
\begin{Highlighting}[]
\CommentTok{\# install.packages("rvest")}
\CommentTok{\# install.packages("tidyverse")}
\CommentTok{\# install.packages("janitor")}
\FunctionTok{library}\NormalTok{(rvest)}
\FunctionTok{library}\NormalTok{(tidyverse)}
\FunctionTok{library}\NormalTok{(janitor)}
\end{Highlighting}
\end{Shaded}

\section{Code}\label{code}

We start off by creating a function to import data from Hockey
Reference.

\begin{Shaded}
\begin{Highlighting}[]
\NormalTok{start\_year }\OtherTok{\textless{}{-}} \DecValTok{1996}
\NormalTok{end\_year }\OtherTok{\textless{}{-}} \DecValTok{2020}

\NormalTok{import\_draft }\OtherTok{\textless{}{-}} \ControlFlowTok{function}\NormalTok{(year)\{}
\NormalTok{  url }\OtherTok{\textless{}{-}} \FunctionTok{str\_glue}\NormalTok{(}\StringTok{"https://www.hockey{-}reference.com/draft/NHL\_\{year\}\_entry.html"}\NormalTok{)}
\NormalTok{  html }\OtherTok{\textless{}{-}} \FunctionTok{read\_html}\NormalTok{(url)}
\NormalTok{  draft\_year\_table }\OtherTok{\textless{}{-}}\NormalTok{ html }\SpecialCharTok{|\textgreater{}} 
    \FunctionTok{html\_element}\NormalTok{(}\StringTok{"table"}\NormalTok{) }\SpecialCharTok{|\textgreater{}} 
    \FunctionTok{html\_table}\NormalTok{() }\SpecialCharTok{|\textgreater{}} 
\NormalTok{    janitor}\SpecialCharTok{::}\FunctionTok{row\_to\_names}\NormalTok{(}\DecValTok{1}\NormalTok{) }\SpecialCharTok{|\textgreater{}} 
\NormalTok{    janitor}\SpecialCharTok{::}\FunctionTok{clean\_names}\NormalTok{()}
\NormalTok{  draft\_year\_table}
\NormalTok{\}}

\FunctionTok{head}\NormalTok{(}\FunctionTok{import\_draft}\NormalTok{(start\_year), }\DecValTok{10}\NormalTok{)}
\end{Highlighting}
\end{Shaded}

\begin{verbatim}
# A tibble: 10 x 21
   overall team    player nat   pos   age   to    amateur_team gp    g     a    
   <chr>   <chr>   <chr>  <chr> <chr> <chr> <chr> <chr>        <chr> <chr> <chr>
 1 1       Ottawa~ Chris~ CA    D     18    2015  Prince Albe~ 1179  71    217  
 2 2       San Jo~ Andre~ RU    D     18    2008  Salavat Yul~ 496   38    82   
 3 3       New Yo~ J.P. ~ CA    RW    18    2011  Val-d'Or Fo~ 822   214   309  
 4 4       Washin~ Alexa~ RU    C     18    2000  Barrie Colt~ 3     0     0    
 5 5       Dallas~ Ric J~ CA    D     18    2007  Soo Greyhou~ 231   19    58   
 6 6       Edmont~ Boyd ~ CA    C     18    2009  Kitchener R~ 627   67    112  
 7 7       Buffal~ Erik ~ US    LW/C  19    2007  Minnesota (~ 545   52    76   
 8 8       Boston~ Johna~ CA    D     18    2004  Medicine Ha~ 44    0     1    
 9 9       Anahei~ Rusla~ BY    D     21    2011  Las Vegas T~ 917   45    159  
10 10      New Je~ Lance~ CA    D     18    2004  Red Deer Re~ 209   4     12   
# i 10 more variables: pts <chr>, x <chr>, pim <chr>, gp_2 <chr>, w <chr>,
#   l <chr>, t_o <chr>, sv_percent <chr>, gaa <chr>, ps <chr>
\end{verbatim}

\bookmarksetup{startatroot}

\chapter{Tidy}\label{tidy}

This is a placeholder for Ch2 -- Tidy.

\bookmarksetup{startatroot}

\chapter{Transform}\label{transform}

This is a placeholder for Ch3 -- Transform.

\bookmarksetup{startatroot}

\chapter{Visualize}\label{visualize}

This is a placeholder for Ch4 -- Visualize.

\bookmarksetup{startatroot}

\chapter{Model}\label{model}

This is a placeholder for Ch5 -- Model.

\bookmarksetup{startatroot}

\chapter{Communicate}\label{communicate}

This is a placeholder for Ch6 -- Communicate.

\bookmarksetup{startatroot}

\chapter*{References}\label{references}
\addcontentsline{toc}{chapter}{References}

\markboth{References}{References}

\phantomsection\label{refs}
\begin{CSLReferences}{0}{1}
\begin{itemize}
\item
  ChatGPT Usage (will add link when done)
\item
  \href{https://summit.sfu.ca/_flysystem/fedora/2023-02/etd22223.pdf}{Valuation
  of NHL Draft Picks using Functional Data Analysis}
\item
  \href{https://soundofhockey.com/2022/06/06/examining-the-value-of-nhl-draft-picks/amp/}{Examining
  the value of NHL Draft picks}
\item
  \href{https://www.broadstreethockey.com/post/nhl-draft-pick-value-trading-up/}{NHL
  draft: What does it cost to trade up?}
\item
  \href{https://www.hockey-reference.com/draft/}{HockeyReference}
\end{itemize}

\end{CSLReferences}

For now this is just a list of links. Later I will formalize it.



\end{document}
